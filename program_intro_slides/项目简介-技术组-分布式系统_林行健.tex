% !Mode:: "TeX:UTF-8"
\documentclass{beamer}

\mode<presentation>
{
\usetheme{Warsaw}

\setbeamercovered{transparent}
}

\usepackage[UTF8]{ctex} % 支持中文

% 分页Slide不编号
\setbeamertemplate{frametitle continuation}{}

% 插入多列
\usepackage{multicol}

% 插入链接
\usepackage{hyperref}
\hypersetup{urlcolor=blue}

% 插入图片
\usepackage{graphicx}

% 插入代码
\usepackage{xcolor}
\definecolor{mygray}{RGB}{245,245,245}

\usepackage{listings}
\lstset{language=Python}
\lstset{escapeinside=``}
\lstset{numbers=left}
\lstset{breaklines}
\lstset{backgroundcolor=\color{mygray}}

% 书签乱码解决方案
% 在文件D:/CTEX/MiKTeX/tex/latex/beamer/base/beamer.cls中找到:
% \PassOptionsToPackage{bookmarks=true,%
%   bookmarksopen=true,%
%   pdfborder={0 0 0},%
%   pdfhighlight={/N},%
%   unicode=true,%       <-------------- 加入此行
%   linkbordercolor={.5 .5 .5}}{hyperref}

% If you have a file called "university-logo-filename.xxx", where xxx
% is a graphic format that can be processed by latex or pdflatex,
% resp., then you can add a logo as follows:

% \pgfdeclareimage[height=0.5cm]{university-logo}{university-logo-filename}
% \logo{\pgfuseimage{university-logo}}



% Delete this, if you do not want the table of contents to pop up at
% the beginning of each subsection:
\AtBeginSubsection[]
{
\begin{frame}<beamer>[plain]
\frametitle{目录}
\begin{multicols}{2}
\tableofcontents[currentsection,currentsubsection]
\end{multicols}
\end{frame}
}

% If you wish to uncover everything in a step-wise fashion, uncomment
% the following command:

%\beamerdefaultoverlayspecification{<+->}

\begin{document}

\title{项目简介-技术组}
\subtitle{cpp-scutter系统简介}

% - Use the \inst{?} command only if the authors have different
%   affiliation.
%\author{F.~Author\inst{1} \and S.~Another\inst{2}}
\author{林行健}

% - Use the \inst command only if there are several affiliations.
% - Keep it simple, no one is interested in your street address.
\institute[Universities of]
{
厦门大学\quad 计算机系}

\renewcommand{\today}{\number\year 年 \number\month 月 \number\day 日}
\date{\today}

% This is only inserted into the PDF information catalog. Can be left
% out.
\subject{Presentations}

% title page
\begin{frame}
\titlepage
\end{frame}

% content
\begin{frame}[plain]
\frametitle{目录}
\begin{multicols}{2}
\tableofcontents %[currentsection,currentsubsection]
\end{multicols}
% You might wish to add the option [pausesections]
\end{frame}

\section{简介}
\subsection{需求分析}
\begin{frame}[allowframebreaks]
\frametitle{需求分析}
\begin{itemize}
\item 运行爬虫
\item 统一调度多台服务器
\item 自动化维护过程
\end{itemize}
\framebreak
% \includegraphics[width=8cm,height=6cm]{}
\end{frame}

\subsection{系统简介}
\begin{frame}
\frametitle{系统简介}
\begin{itemize}
\item 爬虫组件:运行爬虫
\begin{itemize}
  \item 爬虫模块(Crawler):运行爬虫
  \item 代理模块(ProxiesPool):设置HTTP请求头和代理
  \item 日志生成模块(ScutterLog):封装日志模块, 便于调用
\end{itemize}
\item 分布式组件:统一调度多台服务器
\begin{itemize}
  \item 机器调度模块(Work):针对服务器扮演角色进行封装
  \item 消息队列模块(RedisMQ):以redis做中间件, 维护消息队列
  \item 脚本调用模块(ShellServer):封装为shell脚本, 用于linux环境下的调用
\end{itemize}
\item 自动化运维组件:自动化维护过程
\begin{itemize}
  \item 配置模块(config.py+setting.py):参数设置
  \item 自动化运维模块(fabfile.py):批量运维, 一键对多台服务器进行调度
\end{itemize}
\end{itemize}
\end{frame}


\section{爬虫组件}
\subsection{爬虫模块}
\begin{frame}
\frametitle{爬虫模块(Crawler)}
\begin{itemize}
\item 天猫爬虫(TmallScraper)
\begin{itemize}
\item 天猫关键词列表(TmallCategories):基于中国官方2011版CPI分类编制的商品关键词列表
\item 天猫搜索页爬虫(TmallPageScraper.py):抓取天猫搜索页输入关键词后对应搜索结果
\end{itemize}
\item 京东爬虫(JDScraper)
\begin{itemize}
\item 京东关键词列表(JDCategories):类似天猫关键词列表
\item 京东搜索页爬虫(JDPageScraper.py):类似天猫搜索页爬爬虫
\item 京东详情页爬虫(JDScraperDetail.py):抓取京东商品详情页商品ID对应数据
\end{itemize}
\end{itemize}
\end{frame}

\subsection{代理模块}
\begin{frame}
\frametitle{代理模块(ProxiesPool)}
\begin{itemize}
\item 请求头(headers.py):构造HTTP请求报文的头部
\item 代理IP(CheckedProxies):储存的从代理IP网站抓取的代理IP池
\end{itemize}
\end{frame}

\subsection{日志生成模块}
\begin{frame}
\frametitle{日志生成模块(ScutterLog)}
\begin{itemize}
\item 日志生成模块(log.py): 重写Python自带的log模块, 使其更具有灵活性.
\end{itemize}
\end{frame}

\section{分布式组件}

\subsection{机器调度模块}
\begin{frame}
\frametitle{master-slaver分布式系统}
\begin{itemize}
\item 整个系统由一台master, 多台slaver组成
\item master负责生成任务消息, 并将任务消息加入到中间件的消息队列中
\item slaver从消息队列中抓取任务
\end{itemize}
\end{frame}

\begin{frame}
\frametitle{机器调度模块(Work)}
\begin{itemize}
\item master调度(master.py): 用于master机器上的调度, 包括生产者和管理者
\item slaver调度(slaver.py): 用于slaver机器上的调度, 为消费者
\item 爬虫任务分配(distributor.py): 用于分配节点任务
\item 爬虫任务监控(procmon.py): 用于监控爬虫进程是否结束
\end{itemize}
\end{frame}

\subsection{消息队列模块}
\begin{frame}
\frametitle{消息队列}
\begin{itemize}
\item 原理:
\begin{itemize}
\item 基于内存或磁盘的队列
\item 发送消息或者读出消息
\item 提供信息交换
\end{itemize}
\item 功能:
\begin{itemize}
\item 要求服务
\item 交换信息
\item 异步处理
\end{itemize}
\end{itemize}
\end{frame}

\begin{frame}
\frametitle{生产者-消费者模型}
\begin{itemize}
\item 生产者:生产数据,并扔给消息队列
\item 消费者:从消息队列里取数据,并处理数据
\item 消息队列:缓冲区,平衡生产者和消费者的处理能力
\end{itemize}
\end{frame}

\begin{frame}
\frametitle{消息队列模块(RedisMQ)}
\begin{itemize}
\item 建立redis连接(redispool.py):连接redis
\item 管理者(manager.py):统计一天结束后的爬虫情况
\item 消费者(consumer.py):建立消费者模型, 从redis的消息队列中取出url, 进行爬取
\item 生产者(producer.py):建立生产者模型, 生成url, 将url加入redis的消息队列中
\end{itemize}
\end{frame}

\subsection{脚本调用模块}
\begin{frame}
\frametitle{脚本调用模块(ShellServer)}
\begin{itemize}
\item addcron.sh: 添加定时任务脚本
\item compress.sh: 压缩文件脚本
\item init.sh: 初始化脚本
\item kill.sh: 结束爬虫进程脚本
\item manager.sh: 运行管理者脚本
\item procmon.sh: 监控进程脚本
\item repeat.sh: 多次运行爬虫脚本
\item restart.sh: 重启爬虫脚本
\item setlink.sh: 建立log软连接脚本
\item slaver.sh: 运行爬虫脚本
\end{itemize}
\end{frame}

\section{自动化运维组件}
\subsection{配置模块}
\begin{frame}
\frametitle{配置模块(config.py+setting.py)}
\begin{itemize}
\item 不可变配置(config.py):如日期格式, 凭证等
\item 可设置配置(setting.py):如redis地址, 服务器信息等
\end{itemize}
\end{frame}

\subsection{自动化运维模块}
\begin{frame}
\frametitle{自动化运维模块(fabfile.py)}
\begin{itemize}
\item deploy: 批量部署指令
\item restart: 爬虫由于意外中断重启
\item repeat: 多次爬虫
\item kill: 结束爬虫进程
\item add\_cron: 添加定时任务
\end{itemize}
\end{frame}

\begin{frame}
\begin{center}
  \includegraphics[width=7cm,height=7cm]{CPP.jpg}
\end{center}
\end{frame}

\end{document}
