% !Mode:: "TeX:UTF-8"
\documentclass{beamer}

\mode<presentation>
{
\usetheme{Warsaw}

\setbeamercovered{transparent}
}

\usepackage[UTF8]{ctex} % 支持中文

% Table Support
\usepackage{tabu}

% 分页Slide不编号
\setbeamertemplate{frametitle continuation}{}

% 插入多列
\usepackage{multicol}

% 插入链接
\usepackage{hyperref}
\hypersetup{urlcolor=blue}

% 插入图片
\usepackage{graphicx}

% 插入代码
\usepackage{xcolor}
\definecolor{mygray}{RGB}{245,245,245}

\usepackage{listings}
\lstset{language=Python}
\lstset{escapeinside=``}
\lstset{numbers=left}
\lstset{breaklines}
\lstset{backgroundcolor=\color{mygray}}

% 书签乱码解决方案
% 在文件D:/CTEX/MiKTeX/tex/latex/beamer/base/beamer.cls中找到:
% \PassOptionsToPackage{bookmarks=true,%
%   bookmarksopen=true,%
%   pdfborder={0 0 0},%
%   pdfhighlight={/N},%
%   unicode=true,%       <-------------- 加入此行
%   linkbordercolor={.5 .5 .5}}{hyperref}

% If you have a file called "university-logo-filename.xxx", where xxx
% is a graphic format that can be processed by latex or pdflatex,
% resp., then you can add a logo as follows:

% \pgfdeclareimage[height=0.5cm]{university-logo}{university-logo-filename}
% \logo{\pgfuseimage{university-logo}}



% Delete this, if you do not want the table of contents to pop up at
% the beginning of each subsection:
\AtBeginSubsection[]
{
\begin{frame}<beamer>
\frametitle{Outline}
%\begin{multicols}{2}
\tableofcontents[currentsection,currentsubsection]
%\end{multicols}
\end{frame}
}

% If you wish to uncover everything in a step-wise fashion, uncomment
% the following command:

%\beamerdefaultoverlayspecification{<+->}

\begin{document}

\title{项目简介-技术组}
\subtitle{数据库设计}

% - Use the \inst{?} command only if the authors have different
%   affiliation.
%\author{F.~Author\inst{1} \and S.~Another\inst{2}}
\author{黄玺}

% - Use the \inst command only if there are several affiliations.
% - Keep it simple, no one is interested in your street address.
\institute[Universities of]
{
厦门大学\quad 计算机科学系}

\renewcommand{\today}{\number\year 年 \number\month 月 \number\day 日}
\date{\today}

% This is only inserted into the PDF information catalog. Can be left
% out.
\subject{Presentations}

% title page
\begin{frame}
\titlepage
\end{frame}

% content
\begin{frame}
\frametitle{Outline}
%\begin{multicols}{1}
\tableofcontents[currentsection,currentsubsection]
%\end{multicols}
% You might wish to add the option [pausesections]
\end{frame}

\section{The Big Picture of Data Management}

\subsection{Why database management systems?}
\begin{frame}
\frametitle{Data, DB and DBMS}
\begin{description}
\item[Data] is a set of values of qualitative or quantitative variables.
\item[Database] is an organized collection of data containing its schemas, tables, queries, reports, views, and other objects.
\item[DBMS] database management system, which is a computer software application that interacts with the user, other applications, and the database itself and allows the definition, creation, querying, update, and administration of databases. 
\end{description}
\footnotesize *Sometimes a DBMS is loosely referred to as a 'database'.
\end{frame}
\begin{frame}
\frametitle{Databases As a Service}
\begin{table} \small \renewcommand \arraystretch{1.5}
\caption{Table 1-1 Comparing managing data with files and DBs}
\begin{tabu} to \textwidth{X|X|X[2,l]} 
\tabucline-
\textbf{Spec} & \textbf{Files} & \textbf{Databases} \\
\tabucline- 
 Storage & On disk & Managed by database\\
Data status & Raw & Organized \\
Availability & Load on use & Stable\\
Managebility & Messy & Ready for change, backup and administrating\\
Lots of small files & Tricky & Taken care by the impl of storage engine\\ 
Difficulty & Straight but troublesome & Need to learn and also troublesome...for maintainers : )\\ 
\tabucline-
\end{tabu}
\end{table}
\end{frame}

\subsection{Why MongoDB?}
\begin{frame}
\frametitle{Dealing with volume, velocity and variety}
Facts:
\begin{itemize}
  \item Data is becoming semi-structured and unstructured, e.g.\ business transaction and web application data
  \item The scale of data sets is also growing rapidly
  \item Need more powerful systems to support real-time applications
\end{itemize}
MongoDB provides:
\begin{itemize}
  \item Storage based on BSON (the very basis of NoSQL)
  \item Fast ops
  \item Scalability
\end{itemize}
\begin{center}
  \includegraphics[width=4cm,height=1.25cm]{Footages/MongoDB_Logo.png}\\
  \scriptsize Fig. 1-1 MongoDB Logo
 \end{center}
\end{frame}

\section{Designing Database for CPP}

\subsection{Methodology}
\begin{frame}
\frametitle{Methodology}
\begin{itemize}
  \item Deploy MongoDB instances
  \item Design data schemas
  \item Run benchmark to check the performance
  \item Merge with Scrape system
  \item Maintain \& iterate
\end{itemize}\end{frame}
\begin{frame}
\begin{center}
\LARGE Quick Demo
\end{center}
\end{frame}

\subsection{Facts and Challenges}
\begin{frame}
\frametitle{Facts and challenges}
\begin{itemize} \small
  \item The 1G mem of cheap servers cannot meet the need of MongoDB
  \item The disk size of our servers is also limited
  \item Top Level Design: How can we make things easy to use and iterate?
  \begin{itemize}
    \item Monitoring
    \item Automating
    \item Transparency
    \item Decoupling
  \end{itemize}
\end{itemize}
\end{frame}

\subsection{Scale-out:\ From Standalone to Distributed}
\begin{frame}
\frametitle{Solution to hardware limitation:\ Scale-out}
\begin{center}
  \tiny \color{gray} The following figures and texts are adapted from assets of Mining Massive Datasets by\\
  Leskovec, Rajaraman, and Ullman, Stanford University, 2016.\\
   Non-public and non-commerical use only.\\
  \includegraphics[width=4.2cm,height=6cm]{Footages/SNA.png} \\
  \scriptsize \color{black} Fig. 2-1 Single Node Architecture
\end{center}
\end{frame}
\begin{frame}
\frametitle{Solution to hardware limitation:\ Scale-out}
\begin{center}
  \includegraphics[width=10cm,height=5.4cm]{Footages/Cluster.png} \\
  \scriptsize Fig. 2-2 Cluster Architecture
\end{center}
\end{frame}
\begin{frame} \small
\frametitle{Cluster Computing Challenges}
\begin{itemize}
  \item Network bottleneck
  \item Distributed programming
  \item Node Failures
  \begin{itemize}
    \item Hardware breakdown
    \item Software malfunction
    \item Unstable networking
  \end{itemize}
\end{itemize}
Problems:
\begin{itemize}
\item How to store data persistently and keep it available if nodes can fail?
\item How to deal with node failures during a long-running computation?
\item How to hide most of the complexity of the distributed system?
\end{itemize}
\end{frame}
\begin{frame} \small
\frametitle{Redundant Storage Infrastructure}
\begin{itemize}
  \item Distributed File System
  \begin{itemize}
    \item Provides global file namespace, redundancy and availability
    \item E.g.\ Google GFS, Hadoop HDFS
  \end{itemize}
  \item Typical usage pattern
  \begin{itemize}
    \item Huge files (hundreds of GB to TB and even PB)
    \item Data is rarely updated
    \item Reads and appends are common
  \end{itemize}
\end{itemize}
\end{frame}
\begin{frame} \small
\frametitle{Distributed File System}
\includegraphics[width=11.5cm,height=6.2cm]{Footages/DFS.png}
\end{frame}
\begin{frame} \small
\frametitle{Distributed File System}
\includegraphics[width=9.8cm,height=6cm]{Footages/DFSRoles.png}
\end{frame}
\begin{frame} \small
\frametitle{Distributed MongoDB}
Problems:
\begin{itemize}
\item How to store data persistently and keep it available if nodes can fail?
\item How to deal with node failures during a long-running computation?\\
\item How to hide most of the complexity of the distributed system?\\
\end{itemize}
Todo:
\begin{itemize}
\item Auto-sharding
  \begin{itemize}
    \item Replica
    \item Balancing
    \item Primary election
    \item e.t.c.
  \end{itemize}
\end{itemize}
DONE : )
\end{frame}
\begin{frame} \small
\frametitle{Distributed MongoDB}
\begin{center}
\includegraphics[width=10.4cm,height=5.4cm]{Footages/MongoSharding.png}\\
\scriptsize Fig. 2-3 Standalone and Sharding MongoDB Architecture
\end{center}
\end{frame}
\begin{frame}
\begin{center}
\LARGE Any questions?
\end{center}
\end{frame}

\begin{frame}
\begin{center}
  \includegraphics[width=7cm,height=7cm]{Footages/CPP.jpg}
\end{center}
\end{frame}

\end{document}
