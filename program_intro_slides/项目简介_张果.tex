% !Mode:: "TeX:UTF-8"
\documentclass{beamer}

\mode<presentation>
{
\usetheme{Warsaw}

\setbeamercovered{transparent}
}

\usepackage[UTF8]{ctex} % 支持中文

% 分页Slide不编号
\setbeamertemplate{frametitle continuation}{}

% 插入多列
\usepackage{multicol}

% 插入链接
\usepackage{hyperref}
\hypersetup{urlcolor=blue}

% 插入图片
\usepackage{graphicx}

% 插入代码
\usepackage{xcolor}
\definecolor{mygray}{RGB}{245,245,245}

\usepackage{listings}
\lstset{language=Python}
\lstset{escapeinside=``}
\lstset{numbers=left}
\lstset{breaklines}
\lstset{backgroundcolor=\color{mygray}}

% 书签乱码解决方案
% 在文件D:/CTEX/MiKTeX/tex/latex/beamer/base/beamer.cls中找到:
% \PassOptionsToPackage{bookmarks=true,%
%   bookmarksopen=true,%
%   pdfborder={0 0 0},%
%   pdfhighlight={/N},%
%   unicode=true,%       <-------------- 加入此行
%   linkbordercolor={.5 .5 .5}}{hyperref}

% If you have a file called "university-logo-filename.xxx", where xxx
% is a graphic format that can be processed by latex or pdflatex,
% resp., then you can add a logo as follows:

% \pgfdeclareimage[height=0.5cm]{university-logo}{university-logo-filename}
% \logo{\pgfuseimage{university-logo}}



% Delete this, if you do not want the table of contents to pop up at
% the beginning of each subsection:
\AtBeginSubsection[]
{
\begin{frame}<beamer>
\frametitle{目录}
\begin{multicols}{2}
\tableofcontents[currentsection,currentsubsection]
\end{multicols}
\end{frame}
}

% If you wish to uncover everything in a step-wise fashion, uncomment
% the following command:

%\beamerdefaultoverlayspecification{<+->}

\begin{document}

\title{China's Prices Project at Xiamen University\\(CPP@XMU)}
\subtitle{项目简介}

% - Use the \inst{?} command only if the authors have different
%   affiliation.
%\author{F.~Author\inst{1} \and S.~Another\inst{2}}
\author{张果}

% - Use the \inst command only if there are several affiliations.
% - Keep it simple, no one is interested in your street address.
\institute[Universities of]
{
厦门大学\quad 王亚南经济研究院}

\renewcommand{\today}{\number\year 年 \number\month 月 \number\day 日}
\date{\today}

% This is only inserted into the PDF information catalog. Can be left
% out.
\subject{Presentations}

% title page
\begin{frame}
\titlepage
\end{frame}

% content
\begin{frame}
\frametitle{目录}
\begin{multicols}{2}
\tableofcontents
\end{multicols}
% You might wish to add the option [pausesections]
\end{frame}

\section{课题简介}
\subsection{课题背景}
\begin{frame}
\frametitle{课题背景:CPI}
\begin{itemize}
  \item 中国官方CPI存在诸多问题
  \begin{itemize}
    \item 频率低、公布速度慢
    \item 不公布原始数据
    \item 不公布详细编制方案
    \item 官方对于编制方案的解释力度不够
  \end{itemize}
  \item 大数据技术发展迅速
  \begin{itemize}
    \item 海量数据采集:高频采集海量数据
    \item 海量数据储存:管理、实时发布海量数据
    \item 海量数据清洗:处理海量非结构化数据
    \item 海量数据分析:高性能/分布式计算、数据可视化
  \end{itemize}
\end{itemize}
\end{frame}

\begin{frame}
\frametitle{课题背景:线上市场}
\begin{itemize}
  \item 中国互联网市场上,大型活动日(如双十一)越来越多,活动越来越复杂
  \item 大型活动日提供了一个理想的外生冲击,有利于解决内生性问题
  \item 基于结构模型(structural approach)的实证文献非常少
\end{itemize}
\end{frame}

\subsection{课题目标}
\begin{frame}
\frametitle{课题目标}
\begin{itemize}
  \item 第一阶段(2016.2-2016.10):基于官方CPI编制标准和电商实时价格数据,编制线上高频价格指数
  \item 第二阶段(2016.11-present):基于大型活动日和电商微观数据,研究线上市场的产业组织特点
\end{itemize}
\end{frame}

\subsection{文献综述}
\begin{frame}
\frametitle{文献综述:CPI}
\begin{itemize}
  \item 对中国CPI编制的讨论
  \item 基于线上价格数据编制价格指数
  \begin{itemize}
    \item Billion Prices Project at MIT:\url{http://bpp.mit.edu/}
    \item 清数iCPI:\url{http://www.bdecon.com/}
  \end{itemize}
\end{itemize}
\end{frame}

\begin{frame}
\frametitle{文献综述:线上市场}
\begin{itemize}
  \item 消费者:Adda\&Cooper(2006);Dinerstein(2014)
  \item 零售商:Ellison\&Ellison(2005);Fan(2013)
  \item 平台:Rysman(2009)
  \item 互联网数据:Edelman(2012)
\end{itemize}
\end{frame}

\subsection{感兴趣的方向}
\begin{frame}
\frametitle{感兴趣的方向}
\begin{itemize}
\item 主题(Topics)
\begin{itemize}
  \item Dynamic elasticity estimation and welfare analysis
  \item Reputation dynamics
  \item Search obfuscation
\end{itemize}
\item 方法(Methodologies)
\begin{itemize}
  \item 动态结构模型(Dynamic structural approach)
  \item 离散选择模型(Discrete choice model)
\end{itemize}
\end{itemize}
\end{frame}

\subsection{数据来源}
\begin{frame}
\frametitle{数据来源}
\begin{itemize}
  \item 平台选择:市场份额最大的平台——Tmall,JD
  \item 数据来源选择:搜索页搜索结果(前2-5页)$\to$商品详情页
  \item 分类选择:基于CPI分类及其分类解释,分别对不同平台分别编制关键词列表
\end{itemize}
\end{frame}

\subsection{初步成果}
\begin{frame}
\frametitle{初步成果}
\begin{itemize}
  \item 分布式爬虫+自动化运维系统:海量数据采集\footnotemark
  \footnotetext{\tiny 部署需要,尚未开源}
  \item 数据库方案:海量数据管理\footnotemark
  \footnotetext{\tiny \url{https://github.com/xmucpp/cppdbKit}}
  \item 初步数据分析\footnotemark
  \footnotetext{\tiny \url{https://github.com/xmucpp/double11-data}}
  \item 初步资料整理\footnotemark
  \footnotetext{\tiny \url{https://github.com/xmucpp/double11-summary}}
\end{itemize}
\end{frame}


\section{技术简介}
% \subsection{结构模型}
% \subsection{动态离散选择模型}
\subsection{数据采集系统}
\begin{frame}
\frametitle{需求分析}
\begin{itemize}
  \item 可以应对网站反爬的爬虫
  \item 可以调度多台机器的系统
  \item 可以自动化管理多台机器的系统
  \item 可以方便地调用数据的数据库系统
\end{itemize}
\end{frame}

\begin{frame}
\frametitle{解决方案}
\begin{itemize}
  \item 分布式反爬+自动化运维系统
  \begin{itemize}
    \item 爬虫组件:爬虫模块、代理模块、日志模块
    \item 分布式组件:机器调度模块、消息队列模块、脚本调用模块
    \item 自动化运维组件:自动化部署模块、自动化管理模块
  \end{itemize}
  \item 数据库系统(developing)
\end{itemize}
\end{frame}

\section{团队简介}
\subsection{团队成员}
\begin{frame}[allowframebreaks]
\frametitle{团队成员}
\begin{itemize}
  \item 导师:茅家铭老师\footnotemark
  \footnotetext{\tiny
  \url{http://www.wise.xmu.edu.cn/people/faculty/a81c4142-cb73-4f3f-94b4-2b937d4c1acf.html}}
  \item 负责人:张果(14)\footnotemark
  \footnotetext{\tiny \url{https://guo-zhang.github.io/}}
  \item 核心成员:林行健(13)、黄玺(14)、刘晓曼(15)、朱星宇(15)、张祎璘(15)、马宁(16)、唐瀚林(16)
  \item 成员构成:
  \begin{itemize}
    \item 13级本科生:1人
    \item 14级本科生:5人
    \item 15级本科生:7人
    \item 16级本科生:8人
  \end{itemize}
  \end{itemize}
  \end{frame}
  
  \begin{frame}[allowframebreaks]
  \frametitle{团队成员}
  \begin{itemize}
  \item 专业背景:
  \begin{itemize}
    \item 王亚南经济研究院:7人
    \item 经济学院:6人
    \item 计算机系:4人
    \item 管理学院:1人
    \item 外文学院:1人
    \item 人文学院:1人
    \item 国际学院:1人
  \end{itemize} 
  \framebreak
  \begin{figure}
  \begin{center}
  \includegraphics[width=7cm,height=4cm]{cpp-coders.jpg} 
  \caption{CPP主力程序员(从左到右:张果、黄玺、唐瀚林、林行健)}
  \end{center}
  \end{figure}
\end{itemize}
\end{frame}

\subsection{团队分工}
\begin{frame}
\frametitle{团队分工}
\begin{itemize}
  \item 技术部
  \begin{itemize}
    \item 服务器组:林行健(13)、唐瀚林(14)
    \item 数据库组:黄玺(14)、刘理(16)、李蔚然(16)
    \item 爬虫组:马宁(16)、岳忠信(16)
  \end{itemize}
  \item 数据组:张祎璘(15)、周韵丰(14)、吕昕(14)
  \item 宣传组:朱星宇(15)、姜昊(16)、杜雪旸(16)、林逸伦(15)
  \item 文献组:张果(14)、刘晓曼(15)、郝泽栋(15)、庄建伟(15)、张伟贤(16)
  \item 财务组:张晓博(14)、王芷若(15)
\end{itemize}

\end{frame}
\section{进度简介}
\subsection{当前进度}
\begin{frame}
\frametitle{当前进度}
\begin{itemize}
  \item 服务器组:修改分布式爬虫系统的bug
  \item 数据库组:设计、测试数据库方案
  \item 爬虫组:Tmall、JD评论爬虫
  \item 数据组:双十一数据描述统计
  \item 宣传组:项目网站制作
  \item 文献组:双十一、双十二资料整理;梳理文献
\end{itemize}
\end{frame}

\subsection{进度计划}
\begin{frame}
\frametitle{进度计划}
\begin{itemize}
  \item 数据库组:部署数据库
  \item 数据组:双十一、双十二数据描述
  \item 宣传组:上线项目网站
  \item 文献组:双十二资料整理;梳理文献
\end{itemize}
\end{frame}

\subsection{主要困难}
\begin{frame}
\frametitle{主要困难}
\begin{itemize}
  \item 技术:Tmall反爬机制不明,无法对应破解,爬虫效率仍然不太理想
  \item 人员:缺少网络工程师、NLP工程师、机器学习工程师、网站设计、宣传文案等
  \item 硬件:缺少高配置服务器一台
  \item 资金:没有资金来源
\end{itemize}
\end{frame}

\section{联系方式}
\subsection{主要平台}
\begin{frame}[allowframebreaks]
\frametitle{主要平台}
\begin{itemize}
  \item 项目网站(正在修复):\url{http://www.xmucpp.com/}
  \item Github主页:\url{https://github.com/xmucpp}
  \item 知乎专栏:\url{https://zhuanlan.zhihu.com/xmucpp}
  \item 知乎账户:\url{https://www.zhihu.com/people/cpp-45-10}
  \framebreak
  \item 微信公众号:xmucpp2016(XMUCPP)
\end{itemize}
\begin{center}
\includegraphics[width=4.5cm,height=4.5cm]{wechat.jpg}
\end{center}
\end{frame}

\subsection{联系我们}
\begin{frame}
\frametitle{联系我们}
\begin{itemize}
  \item 项目邮箱(张果):zhangguocpp@163.com
  \item 加入我们(刘晓曼):liuxiaomancpp@163.com
  \item 知乎:
  \begin{itemize}
    \item CPP:\url{https://www.zhihu.com/people/cpp-45-10}
    \item 张果:\url{https://www.zhihu.com/people/zhang_guo}
    \item 刘晓曼:\url{https://www.zhihu.com/people/liu-xiao-man-3-2}
    \item 朱星宇:\url{https://www.zhihu.com/people/felix-zhu-23}
  \end{itemize}
\end{itemize}
\end{frame}

\begin{frame}
\begin{center}
  \includegraphics[width=7cm,height=7cm]{CPP.jpg}
\end{center}
\end{frame}

\end{document}
