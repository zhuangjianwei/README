% !Mode:: "TeX:UTF-8"
\documentclass{beamer}

\mode<presentation>
{
\usetheme{Warsaw}

\setbeamercovered{transparent}
}

\usepackage[UTF8]{ctex} % 支持中文

% 分页Slide不编号
\setbeamertemplate{frametitle continuation}{}

% 插入多列
\usepackage{multicol}

% 插入链接
\usepackage{hyperref}
\hypersetup{urlcolor=blue}

% 插入图片
\usepackage{graphicx}
 
% 插入代码
\usepackage{xcolor}
\definecolor{mygray}{RGB}{245,245,245}

\usepackage{listings}
\lstset{language=Python}
\lstset{escapeinside=``}
\lstset{numbers=left}
\lstset{breaklines}
\lstset{backgroundcolor=\color{mygray}}

% 书签乱码解决方案
% 在文件D:/CTEX/MiKTeX/tex/latex/beamer/base/beamer.cls中找到:
% \PassOptionsToPackage{bookmarks=true,%
%   bookmarksopen=true,%
%   pdfborder={0 0 0},%
%   pdfhighlight={/N},%
%   unicode=true,%       <-------------- 加入此行
%   linkbordercolor={.5 .5 .5}}{hyperref}

% If you have a file called "university-logo-filename.xxx", where xxx
% is a graphic format that can be processed by latex or pdflatex,
% resp., then you can add a logo as follows:

% \pgfdeclareimage[height=0.5cm]{university-logo}{university-logo-filename}
% \logo{\pgfuseimage{university-logo}}



% Delete this, if you do not want the table of contents to pop up at
% the beginning of each subsection:
\AtBeginSubsection[]
{
\begin{frame}<beamer>
\frametitle{目录}
\begin{multicols}{2}
\tableofcontents%[currentsection,currentsubsection]
\end{multicols}
\end{frame}
}

% If you wish to uncover everything in a step-wise fashion, uncomment
% the following command:

%\beamerdefaultoverlayspecification{<+->}

\begin{document}

\title{项目简介-爬虫组}
\subtitle{爬虫实例:JD评论爬虫}

% - Use the \inst{?} command only if the authors have different
%   affiliation.
%\author{F.~Author\inst{1} \and S.~Another\inst{2}}
\author{岳忠信}

% - Use the \inst command only if there are several affiliations.
% - Keep it simple, no one is interested in your street address.
\institute[Universities of]
{
厦门大学\quad 经济学院}

\renewcommand{\today}{\number\year 年 \number\month 月 \number\day 日}
\date{\today}

% This is only inserted into the PDF information catalog. Can be left
% out.
\subject{Presentations}

% title page
\begin{frame}
\titlepage
\end{frame}

% content
\begin{frame}
\frametitle{目录}
\begin{multicols}{2}
\tableofcontents[currentsection,currentsubsection]
\end{multicols}
% You might wish to add the option [pausesections]
\end{frame}

\section{通用爬虫设计}
\subsection{爬虫主要流程:请求,解析,储存}

\begin{frame}
\frametitle{请求}
\begin{itemize}
  \item 判断静态还是动态页面。\\ 方法:审查元素(查看源代码)看我们所需的信息是否出现。
  \item 库的选择:通常请求用requests库。
  \item 具体请求方式:get或者post(视具体情况而定)
\end{itemize}
\end{frame}

\begin{frame}
\frametitle{解析}
\begin{itemize}
  \item 库的选择:bs4或者json
  \item 方法:静态——BeautifulSoup, 动态——json
\end{itemize}
\end{frame}

\begin{frame}
\frametitle{储存}
\begin{itemize}
   \item 储存格式:csv或者json 
   \item 库的选择:Python的csv模块
\end{itemize}
\end{frame}

\section{爬虫实例:JD评论爬虫}
\subsection{静态爬虫:以京东的产品价格为例}
\begin{frame}
\frametitle{步骤}
\begin{enumerate}
  \item 判断出它是静态内容
  \item 利用requests库请求html
  \begin{itemize}
  	\item 反爬
  	\item 具体症状:返回的html错误或者出现403等报错数字
  	\item
  	原因:一般,浏览器在向服务器发送请求的时候,会有一个请求头——User-Agent,它用来标识浏览器的类型.当我们使用requests来发送请求的时候,默认的User-Agent是python-requests/2.**。
    \item 反爬策略:修改User-Agent如修改为'Mozilla/5.0 (Macintosh; Intel Mac OS X
    10\_11\_2) AppleWebKit/537.36 (KHTML, like Gecko) Chrome/47.0.2526.80 Safari/537.36'
  \end{itemize}
  \item 请求下正常的html之后,用BeautifulSoup进行解析。
\end{enumerate}
\end{frame}

\subsection{动态爬虫:以京东的商品评论为例}
\begin{frame}
难点:
\begin{itemize}
  \item 如何获取url
  \item 如何请求数据
  \item 如何处理返回的html
\end{itemize}
\end{frame}

\begin{frame}
\frametitle{步骤}
\begin{enumerate}
  \item 审查元素(建议使用谷歌浏览器)\\注意:刷新,浏览整个页面\\观察内容:一般在xhr或者js中,获取正确的url
  \item 获取内容 \\ 使用库:requests \\ 具体方法:
  \begin{itemize}
  	\item 观察headers,决定使用get或post以及对应的请求头\\(一般来讲,其中我们需要关注的是Cookie / Host / Origin
  	/ Referer / User-Agent / X-Requested-With等参数。)
  	\item 好消息是,这样的请求往往得到的内容是json格式的,所以我们非但不会加重爬虫的任务,反而可能会省去解析HTML的功夫。
  	\item 运用json库将其封装成字典
  \end{itemize}
  
\end{enumerate}
接下来我们就可以利用对python的list和dict的操作来获取我们想要的信息
\end{frame}

\subsection{常见问题}
\begin{frame}
\begin{enumerate}
  \item 获取html时出现错误\\解决方案:观察headers\\判断是否是反爬(print出来状态码)解决方法:修改请求头
  \item 解析html的坑:
  \begin{itemize}
  	\item 返回的不是正常的json格式\\方法:import re 用正则表达式洗字符串
  	\item 编码错误\\查看编码:观察headers或print出来\\方法:转化成unicode.
  	\end{itemize}
  	
\end{enumerate}
解决了这些问题,我们就应该能顺利地写出一篇单面爬虫了,那么如何翻页呢?
\end{frame}

\subsection{翻页策略:构造url}
\begin{frame}
\begin{enumerate}
  \item 观察url,修改其中的参数来重新运行观察有什么效果
  \item 找到最大面数,然后利用字符串格式化来修改url\\用迭代的方法写成循环。
  \item 观察运行效果
\end{enumerate}
\end{frame}

\subsection{常见错误}
\begin{frame}
\begin{enumerate}
  \item 返回的是空list现在猜测原因是正则表达式的问题)
  \item 编码错误(解决思路:从一开始请求时就将其转化为unicode)
  \item{ValueError}
\end{enumerate}
\end{frame}

\subsection{解决新思路}
\begin{frame}
\end{frame}

\subsection{爬虫的修改}
\begin{frame}
\begin{enumerate}
  \item{函数式编程}
  \item{循环的写法}
  \item{报错处理}
\end{enumerate}
\end{frame}



\begin{frame}
\begin{center}
  \includegraphics[width=7cm,height=7cm]{CPP.jpg}
\end{center}
\end{frame}

\end{document}
